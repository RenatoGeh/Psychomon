\documentclass[a4paper]{article}

\usepackage[english]{babel}
\usepackage[utf8]{inputenc}

\title{Relatório de Projeto - Psychomon}

\author{
Guilherme Schützer - NUSP 8658544 \\
Renato Geh         - NUSP 8536030 \\
Ricardo Lira       - NUSP 8536131 \\
Yan Couto          - NUSP 8536023
}

\date{}

\begin{document}
\maketitle


\section{Primeira Fase}

\begin{verbatim}
                          miau, eh isso ai
                            .-. \_/ .-.
                            \.-\/=\/.-/
                         '-./___|=|___\.-'
                        .--| \|/`"`\|/ |--.
                       (((_)\  .___.  /(_)))
                        `\ \_`-.   .-'_/ /`_
                          '.__       __.'(_))
                              /     \     //
                             |       |__.'/
                             \       /--'`
                         .--,-' .--. '----.
                        '----`--'  '--`----'
\end{verbatim}

 Na primeira fase do projeto, primeiramente planejamos o que fariamos e
começamos a primeira parte do diagrama de classes. Após debater e discutir como
cada parte desa fase iria se interagir, e com o diagrama feito, ficou mais fácil
a programação e implementação das ideias.

    Inicialmente fizemos a base do programa, incluindo a definição de um
pokemon, seus atributos, seus ataques, o que é um ataque, e como estes
seriam usados. Ainda sem implementar as fórmulas de combate ou outras funções
mais específicas, criamos duas classes que auxiliariam em organizar as
batalhas e dados para o programa. A classe "Pokedex" serve para "coletar" e
armazenar os pokemons e seus respectivos atributos/ataques, enquanto a classe
"Pokestadium" tem como objetivo ser um criador de batalhas, que organiza quais
pokemons lutam, e inicializa o combate.

    Com essa base já preparada, começamos a planejar as lutas ("pokebattles"),
que já teriam uma interface simples implementada para o usuário, de forma que
seja visível o combate entre dois pokemons. A classe "Pokebattle" então
definiria quem começaria (o pokemon mais rapido), e organizaria os turnos de
cada pokemon, mostrando os movimentos possíveis e imprimindo alguns de seus
atributos e do seu oponente.

    Implementar o dano foi um pouco trabalhoso, a fórmula não é tão simples
quanto pensávamos, mas tirando escrever toda a tabela de type effectiveness 
o resto foi bem rápido. Como já tinhamos deixado preparado o cálculo do ataque
antes, não foi necessário mudar nada fora da classe ataque, o que ajudou bastante.

    Apesar de vários testes terem sido feitos durante todo esse processo, ao
terminar o sistema de batalha mais simples, começamos a criar pokemons com a
finalidade de testar o combate. Essa criação inicial foi feita "criativamente",
inventando nomes e atributos aleatórios, sem se basear em pokemons existentes.
Conforme fossemos arrumando alguns erros e bugs, começamos a pegar dados de
sites como bulbapedia.com para nossos pokemons, adaptando as informações ao
formato que nosso programa aceita como input.

\end{document}