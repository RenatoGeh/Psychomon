\documentclass[a4paper]{article}

\usepackage[english]{babel}
\usepackage[utf8]{inputenc}
\usepackage{mathtools}

\title{Relatório de Projeto - Psychomon}

\author{
Guilherme Schützer - NUSP 8658544 \\
Renato Geh         - NUSP 8536030 \\
Ricardo Lira       - NUSP 8536131 \\
Yan Couto          - NUSP 8536023
}

\date{}

\begin{document}
\maketitle

\section*{Instruções}
Para rodar o projeto basta usar
\begin{center}
\texttt{python3 Psychomon.py}
\end{center}
O programa apresentará um menu com as opções, que são:
\begin{description}
\item[Read Pokémon] \hfill \\ Lê o Pokémon como está no enunciado
\item[List Pokémon] \hfill \\ Lista o nome de todos os Pokémons já lidos
\item[Battle!] \hfill \\ Começa uma batalha com Pokémons que já foram lidos
(os nomes deles são usados como seus identificadores) \\ Não funciona caso
nenhum Pokémon tenha sido inserido
\item[Open Server] \hfill \\ Inicializa o Modo Servidor, é necessário utilizar
os Pokémons já inseridos para lutar, como em \textbf{Battle!}
\item[Open Client] \hfill \\ Inicializa o Modo Cliente, é necessário escolher um
Pokémon que já foi lido, como em \textbf{Battle!}
\item[Quit] \hfill \\ Termina o programa
\end{description}
Para não ter o trabalho de inserir os Pokémons novamente toda vez que quisessemos
rodar o projeto, criamos um arquivo que já tem alguns Pokémons feitos, o \texttt{Pokefile}.
Para usá-lo basta rodar
\begin{center}
\texttt{cat input/Pokefile - | python3 Psychomon.py}
\end{center}
Note que algum texto "inutil" aparecerá no terminal pois o programa imprime a
interação com o usuário mesmo quando está lendo a entrada do arquivo, mas depois
disso já existirão 3 Pokémons lidos, basta usar \textbf{List Pokémon} para saber
seus nomes e começar a usá-los em batalhas.
\newpage
Para testar o programa, basta usar
\begin{center}
\texttt{python3 Poketest.py}
\end{center}
O programa \texttt{Poketest} executa todos os testes que estão na pasta \texttt{tests},
que usam o pacote \emph{unittest}.

\section{Primeira Fase}

\begin{verbatim}
                          Miau, é isso ai
                            .-. \_/ .-.
                            \.-\/=\/.-/
                         '-./___|=|___\.-'
                        .--| \|/`"`\|/ |--.
                       (((_)\  .___.  /(_)))
                        `\ \_`-.   .-'_/ /`_
                          '.__       __.'(_))
                              /     \     //
                             |       |__.'/
                             \       /--'`
                         .--,-' .--. '----.
                        '----`--'  '--`----'
\end{verbatim}

 Na primeira fase do projeto, primeiramente planejamos o que fariamos e
começamos a primeira parte do diagrama de classes. Após debater e discutir como
cada parte desa fase iria se interagir, e com o diagrama feito, ficou mais fácil
a programação e implementação das ideias.

    Inicialmente fizemos a base do programa, incluindo a definição de um
Pokémon, seus atributos, seus ataques, o que é um ataque, e como estes
seriam usados. Ainda sem implementar as fórmulas de combate ou outras funções
mais específicas, criamos duas classes que auxiliariam em organizar as
batalhas e dados para o programa. A classe "Pokedex" serve para "coletar" e
armazenar os pokemons e seus respectivos atributos/ataques, enquanto a classe
"Pokestadium" tem como objetivo ser um criador de batalhas, que organiza quais
pokemons lutam, e inicializa o combate.

    Com essa base já preparada, começamos a planejar as lutas ("pokebattles"),
que já teriam uma interface simples implementada para o usuário, de forma que
seja visível o combate entre dois pokemons. A classe "Pokebattle" então
definiria quem começaria (o pokemon mais rapido), e organizaria os turnos de
cada pokemon, mostrando os movimentos possíveis e imprimindo alguns de seus
atributos e do seu oponente.

    Implementar o dano foi um pouco trabalhoso, a fórmula não é tão simples
quanto pensávamos, mas tirando escrever toda a tabela de type effectiveness 
o resto foi bem rápido. Como já tinhamos deixado preparado o cálculo do ataque
antes, não foi necessário mudar nada fora da classe ataque, o que ajudou bastante.

    Apesar de vários testes terem sido feitos durante todo esse processo, ao terminar o sistema de batalha mais simples, começamos a criar Pokémons com a
finalidade de testar o combate. Essa criação inicial foi feita "criativamente",
inventando nomes e atributos aleatórios, sem se basear em pokemons existentes.
Conforme fossemos arrumando alguns erros e bugs, começamos a pegar dados de
sites como bulbapedia.com para nossos pokemons, adaptando as informações ao
formato que nosso programa aceita como input.

\newpage
\section{Segunda Fase}

\begin{verbatim}
                       ."-,.__
                       `.     `.  ,
                    .--'  .._,'"-' `.
                   .    .'         `'
                   `.   /          ,'
                     `  '--.   ,-"'           Crescer, Conquistar
                      `"`   |  \
                         -. \, |
                          `--Y.'      ___.
                               \     L._, \
                     _.,        `.   <  <\                _
                   ,' '           `, `.   | \            ( `
                ../, `.            `  |    .\`.           \ \_
               ,' ,..  .           _.,'    ||\l            )  '".
              , ,'   \           ,'.-.`-._,'  |           .  _._`.
            ,' /      \ \        `' ' `--/   | \          / /   ..\
          .'  /        \ .         |\__ - _ ,'` `        / /     `.`.
          |  '          ..         `-...-"  |  `-'      / /        . `.
          | /           |L__           |    |          / /          `. `.
         , /            .   .          |    |         / /             ` `
        / /          ,. ,`._ `-_       |    |  _   ,-' /               ` \
       / .           \"`_/. `-_ \_,.  ,'    +-' `-'  _,        ..,-.    \`.
        '         .-f    ,'   `    '.       \__.---'     _   .'   '     \ \
      ' /          `.'    l     .' /          \..      ,_|/   `.  ,'`     L`
      |'      _.-""` `.    \ _,'  `            \ `.___`.'"`-.  , |   |    | \
      ||    ,'      `. `.   '       _,...._        `  |    `/ '  |   '     .|
      ||  ,'          `. ;.,.---' ,'       `.   `.. `-'  .-' /_ .'    ;_   ||
      || '              V      / /           `   | `   ,'   ,' '.    !  `. ||
      ||/            _,-------7 '              . |  `-'    l         /    `||
       |          ,' .-   ,' ||               | .-.        `.      .'     ||
       `'        ,'    `".'    |               |    `.        '. -.'       `'
                /      ,'      |               |,'    \-.._,.'/'
                .     /        .               .       \    .''
              .`.    |         `.             /         :_,'.'
                \ `...\   _     ,'-.        .'         /_.-'
                 `-.__ `,  `'   .  _.>----''.  _  __  /
                      .'        /"'          |  "'   '_
                     /_|.-'\ ,".             '.'`__'-( \
                       / ,"'"\,'               `/  `-.|"
\end{verbatim}

  A segunda fase do projeto foi bem diferente. Já tinhamos toda a base da primeira fase
feita, o grande desafio foi aprender a usar novas bibliotecas para conseguir fazer tudo
aquilo funcionar 'online'. Tinhamos pouca familiaridade com XML, então foi por ali que 
começamos, já que o servidor e o cliente precisavam daquilo para funcionar.

   Depois de pesquisar algumas bibliotecas e tutoriais de como criar, ler e validar XML's
em Python, decidimos usar a \emph{lxml}, que também permite a validação dos XML's usando
um xsd, que ao avançar no projeto, se mostrou muito útil para encontrar erros, especialmente quando
começamos a criar o servidor e o cliente.

   Porém, tivemos uma grande dificuldade de usar essa biblioteca no Windows, tentando até
mesmo compilar o seu código fonte, sem sucesso. Resolvemos portanto desenvolver o projeto no
Linux, onde está biblioteca já vem instalada por padrão. 

    Tirando essa dificuldade inicial, o resto de XML's não foi muito trabalhoso, apenas
esforço braçal.

   Depois de pronta essa parte, já conseguimos um programa que cria XML's a partir de
Pokemons e Pokemons a partir de XML's. Então começamos a criação da interface servidor/cliente.
Para o servidor usamos a biblioteca \emph{flask}, que tornou bem fácil a criação de um servidor
funcional. O fluxo da batalha foi um pouco diferente do normal, por isso tivemos que adaptar
parte do programa para isso. Para o cliente usamos \emph{requests}, e o processo foi bem parecido
com a batalha normal.

   A interação entre os dois demorou um pouco para ficar boa, ocorreram algumas dificuldades
para organizar o fluxo de dados, mas com um pouco de tempo conseguimos resolver. A primeira batalha
usando dois terminais diferentes foi bem divertida.

   Um 'problema' que encontramos é que fazendo as coisas exatamente como é dito no enunciado
do EP, não é possível saber o HP inicial e PP's iniciais dos ataques da parte do servidor (a 
menos que \emph{gambiarras} sejam utilizadas) e muito menos da parte do cliente. Além disso, o
cliente fica sem nenhuma indicação de quais ataques foram usados pelo servidor e quanto dano
os ataques causaram (além das informações adicionais).

   Mudamos os testes para um diretório apropriado. Para isso tivemos de aprender a usar o
sistema de módulos de python e como funciona para se dar "import" em subdiretórios.

   Por último, também tivemos que arrumar o idioma do EP, já que em partes a comunicação com
o usuário se dava em inglês e outras em português. Decidimos mudar, por fim, tudo para o inglês.

\newpage
\section{Terceira Fase}

\begin{verbatim}
                                                   _,'|
                                                 .'  /
                        __                     ,'   '
                       `  `.                 .'    '
                        \   `.             ,'     '       A inteligência suprema
                         \    `.          ,      /
                          .     `.       /      ,
                          '       ..__../'     /
                           \     ,"'   '      . _.._
                            \  ,'             |'    `"._
                             |/               ,---.._   `.
                           ,-|           .   '       `-.  \
                         ,'  |     ,   ,'   :           '__\_
                         |  /,_   /  ,U|    '            |   .__
                         `,' `.\ `./..-'  __ \           |   `. `.
                           `",_|  /     ,"  `.`._       .|     \ |
                          / /_.| j  ---'.     `._`-----`.`     | |
                         / // ,|`'  `-/' `.      `"/-+--'    ,'  `.
                     _,.`,'| / |.'  -,' \  \       \ '._    /     |
     .--.      _,.-"'   `| L \ \__ ,^.__.\  `.  _,--`._,>+-'  __,-'
    :    \   ,'          |  | \          /.   `'      '.  `--'| \
    '    | ,-.. `'   _,--' ,'  \        `.\            7      |,.\
     `._ '.  .`.    .>  `-.-    |-.""---..-\        _>`       `.-'
        `.,' | l  ,' ,>         | `.___,....\._    ,--``-.
       j | .'|_|.'  /_         /   _|         \`"--+--.   ` ,..._
       |_`-'/  |     ,' ,.._,.'"""'\           `--'    `-..'     `".
         "-'_,+'\    '^-     |      \                    /         |
              |_/         __ \       .                   `.`.._  ,'`.
                      _.:'__`'        `,.                  |   `'   |
                     `--`-..`"        /--`               ,-`        |
                       `---'---------'                   ""| `#     '.
                                                           `._,       `:._
                                                             `|   ,..  |  '.
                                                             j   '.  `-+---'
                                                             |,.. |
                                                              `. `;
                                                                `' 
\end{verbatim}

A última fase foi bem mais simples que as outras. Nesta, bastava a "Inteligência Artificial"
de nosso programa escolher os ataques sozinhos quando usando o Modo Cliente ou Servidor.

Como o jeito que os ataques foram implementados foi uma tarefa bem simples, pois como os
ataques só dão dano direto e não causam efeitos secundários, e não é possível mudar de
pokemon durante a batalha, não existe necessidade de \emph{economizar} PPs para usar
em batalhas futuras, basta apenas escolher o ataque disponível que tem maior "dano médio".

A fórmula para o "dano médio" que definimos foi:
$$ \text{dano médio} = \text{chance acerto} \times (1 + \text{chance crítico}) \times \text{dano base} $$
Onde \emph{dano base} é o dano do ataque, sem contar o crítico e o modificador aleatório.

Com isso feito, focamos em melhorar o programa em uma visão maior, aperfeiçoando um pouco
os testes e o código onde achamos necessário. Novamente, utilizamos essa última parte do
projeto para olhar nosso programa de forma mais abrangente, analisando melhor como cada parte
interage entre sí, tentando aprimorar o que possível.

\newpage
\section{Comentários Finais}

\begin{verbatim}


\end{verbatim}

Como comentário do grupo em relação à essa última parte do projeto, achamos que testar o programa
contra o de outros grupos não parece que vai dar muito certo, pois existe uma chance considerável
de os programas não serem compatíveis um com o outro. Isso devido a algumas coisas não ficaram muito
padronizadas no enunciado, como a ordem que os Pokémons tem que ficar no XML (nosso grupo assumiu
que o Pokémon do Cliente fica sempre na primeira posição e o do Servidor na segunda).
Além disso, como os ataques são um tanto simples, sem ter a parte interessante (e desafiante) de uma
batalha Pokemon com vários modificadores e ataques variados, a vitória vai se inclinar fortemente
nos fatores aleatórios (modificador aleatório e críticos) e nos Pokémons escolhidos (que podem
ter valores arbitrarios de atributos se não forem balanceados para todos os grupos).

Apesar desse pormenor, o projeto teve seus estimulos, especialmente no começo, no qual tivemos que
aprender a fazer nosso primeiro "grande" projeto em Python, utilizando dos conhecimentos de Patterns
e outras ferramentas que aprendemos no curso, quanto na parte de aprender e trabalhar com bibliotecas
pra transformar nosso programa em algo online.

Entendemos que o curso foi em boa parte experimental, e por isso teve seus contraventos, mas a ideia
de fazer um Pokemon acabou sendo bem divertida, e tem potencial para, caso em futuros semestre repita
tal projeto, de ser algo maior e mais "aventuroso" para os alunos. 

\end{document}
